%-------------------------------------------------------------------------------
%	SECTION TITLE
%-------------------------------------------------------------------------------
\cvsection{Education}


%-------------------------------------------------------------------------------
%	CONTENT
%-------------------------------------------------------------------------------
\begin{cventries}

%---------------------------------------------------------
\cventry
{M.Sc. in Computer Science - AI curriculum} % Degree
{University of Pisa} % Institution
{Pisa, Italy} % Location
{Sep. 2020 - Dec. 2023} % Date(s)
{
    \begin{cvitems} % Description(s) bullet points
        \item{\textbf{Final mark:} 110/110 \textit{cum Laude}.}
        \item{\textbf{Thesis:} \textit{Continual Learning for Non-Autoregressive Neural Machine Translation}.\\
        The task involved working with two non-autoregressive models, namely CMLM and GLAT, and evaluating their performance in a multilingual scenario.The objective was to assess how well they address the issue of catastrophic forgetting while employing the continual strategy of experience replay.}
    \end{cvitems}
}

\cventry
{B.Sc. in Computer Science}
{University of Florence}
{Florence, Italy}
{Sep. 2016 - Feb. 2020}
{
    \begin{cvitems}
        \item{\textbf{Final mark:} 110/110.}
        \item{\textbf{Thesis:} \textit{Genetic Algorithms and their Applications}.\\
        The behaviour of genetic algorithms was analyzed and their advantages and disadvantages were also evaluated. The algorithms were first applied to a simple case of function maximization and then utilised in a path-finding context.}
    \end{cvitems}
}

\cventry
{High School} % Degree
{ISISTL Russell-Newton} % Institution
{Scandicci, Italy} % Location
{Sep. 2011 - Jul. 2016} % Date(s)
{
    \begin{cvitems}
        \item{\textbf{Final mark:} 100/100 \textit{cum Laude.}}
        \item{Took part in the first edition of \textit{Progetto TRIO} (an intership that spans over an entire school-year) during my fourth year.}
    \end{cvitems}
}

%---------------------------------------------------------
\end{cventries}
