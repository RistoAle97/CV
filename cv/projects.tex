%-------------------------------------------------------------------------------
%	SECTION TITLE
%-------------------------------------------------------------------------------
\cvsection{University Projects}


%-------------------------------------------------------------------------------
%	CONTENT
%-------------------------------------------------------------------------------
\begin{cventries}

%---------------------------------------------------------
\cventry
{Employment of rule-based and minimax strategies for a Pokémon battle bot} % Project Description
{Artificial Intelligence Fundamentals} % Course
{} % Location
{Oct. 2022 - Jan. 2023} % Date(s)
{
    \begin{cvitems}
        \item{Built a Pokémon battle bot that can be challenged via an online simulator. My main contribution was the damage and stats calculation, while also defining the rules for the rule-based bot.}
        \item{\textbf{Technical skills:} Python, Git.}
    \end{cvitems}
}

%---------------------------------------------------------
\cventry
{Application of various optimization algorithms to a linear least squares problem} % Project Description
{Computational Mathematics for Learning and Data Analysis} % Course
{} % \href{https://github.com/nikodallanoce/ComputationalMathematics}{\faGithubSquare\acvHeaderIconSep\@Project repository}} % Location
{Jun. 2022 - Sep. 2022} % Date(s)
{
    \begin{cvitems}
        \item{Applied L-BFGS, Thin QR factorization, Conjugate Gradient and Gradient Descent to solve a linear least squares problem with an ill-conditioned matrix.}
        \item{\textbf{Technical skills:} MATLAB.}
    \end{cvitems}
}

%---------------------------------------------------------
\cventry
{Pathway analysis of disease's proteins} % Project Description
{Computational Health Laboratory} % Course
{} % Location
{Apr. 2022 - May 2022} % Date(s)
{
    \begin{cvitems}
        \item{Built a protein-to-protein graph starting from a single protein that is the main responsible for a disease and found the correlation with all the other proteins in the network.}
        \item{\textbf{Technical skills:} Python (pandas, seaborn), Jupyter Notebook, Git.}
    \end{cvitems}
}

%---------------------------------------------------------
\cventry
{Simulation of highway traffic via cellular automata} % Project Description
{Computational Models for Complex Systems} % Course
{} % Location
{May 2022 - May 2022} % Date(s)
{
    \begin{cvitems}
        \item{Simulated the flow of highway traffic in different scenarios using the paradigm of cellular automata.}
        \item{\textbf{Technical skills:} Java, Git.}
    \end{cvitems}
}

%---------------------------------------------------------
\cventry
{Parallelization of a custom KNN algorithm} % Project Description
{Parallel and Distributed Systems: Paradigms and Models} % Course
{} % Location
{Jan. 2022 - Feb. 2022} % Date(s)
{
    \begin{cvitems}
        \item{I used the standard library of C++ and the FastFlow library to parallelize a custom implementation of the KNN algorithm and, then, I compared the performance of both implementations.}
        \item{\textbf{technical skills:} C++, Python, Linux.}
    \end{cvitems}
}

%---------------------------------------------------------
\cventry
{Object recognition for an autonomous driving car} % Project Description
{Smart Applications} % Course
{} % Location
{Nov. 2021 - Jan. 2022} % Date(s)
{
    \begin{cvitems}
        \item{Took part in a team of eight people that was tasked in developing an object recognition model for both the stereocamera and lidar of an autonomous driving vehicle.}
        \item{\textbf{Technical skills:} Python (pandas, YOLOv3), Git.}
    \end{cvitems}
}

%---------------------------------------------------------
\cventry
{Analysis and study of tennis matches data} % Project Description
{Data Mining} % Course
{} % Location
{Sep. 2021 - Jan. 2022} % Date(s)
{
    \begin{cvitems}
        \item {I worked on a dataset of more than 100k matches: first of all I had to clean it from missing or wrong data. Then, I gathered info on all the players and classified them into different categories by analyzing their performance. Finally, I displayed the results in a human-readable way.}
        \item{\textbf{Technical skills:} Python (pandas, seaborn, numpy, scikit), Jupyter Notebook, Git.}
    \end{cvitems}
}

%---------------------------------------------------------
\cventry
{Comparison of different NMT models} % Project Description
{Human Language Technologies} % Course
{} % Location
{Sep. 2021 - Dec. 2021} % Date(s)
{
    \begin{cvitems}
        \item{Compared different Neural Machine Translation models in order to attest their performances while changing their decoder.}
        \item{\textbf{Technical skills:} Python (Tensorflow), Jupyter Notebook, Git.} 
    \end{cvitems}
}

%---------------------------------------------------------
\cventry
{Telegram bot for monitoring room temperature} % Project description
{Mobile and Cyber-Physical Systems} % Course
{} % Location
{Mar. 2021 - Jun. 2021} % Date(s)
{
    \begin{cvitems}
        \item{Developed a Telegram bot that updates the user with data from some chosen rooms. This was done in order to let a shop owner know if its refrigerator cells are having an issue and to assess if such rooms are complying with the HACCP regulamentations.}
        \item{\textbf{Technical skills:} Python, MongoDB, Java, Git}
    \end{cvitems}
}

%---------------------------------------------------------
\cventry
{Development of a neural network from scratch} % Project description
{Machine Learning} % Course
{} % Location
{Nov. 2020 - Jan. 2021} % Date(s)
{
    \begin{cvitems}
        \item{Built a simple library that deals with the construction of a neural network's layers while also implementing the forward and backward passes.}
        \item{\textbf{Technical skills:} Python (numpy), Git.}
    \end{cvitems}
}

%---------------------------------------------------------
\end{cventries}
