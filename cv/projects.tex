%-------------------------------------------------------------------------------
%	SECTION TITLE
%-------------------------------------------------------------------------------
\cvsection{University Projects}


%-------------------------------------------------------------------------------
%	CONTENT
%-------------------------------------------------------------------------------
\begin{cventries}

%---------------------------------------------------------
\cventry
{Employment of rule-based and minimax strategies for a Pokémon battle bot} % Project Description
{Artificial Intelligence Fundamentals} % Course
{\href{https://github.com/nikodallanoce/PokeBOT}{\faGithubSquare Project repository}} % Location
{Oct. 2022 - Jan. 2023} % Date(s)
{
    \begin{cvitems}
        \item{Built a Pokémon battle bot that can be challenged via an online simulator.The main contribution was the calculation of damage and stats, as well as the definition of rules for the rule-based bot.}
        \item{\textbf{Technical skills:} Python, Git.}
    \end{cvitems}
}

%---------------------------------------------------------
\cventry
{Application of various optimization algorithms to a linear least squares problem} % Project Description
{Computational Mathematics for Learning and Data Analysis} % Course
{\href{https://github.com/nikodallanoce/ComputationalMathematics}{\faGithubSquare Project repository}} % Location
{Jun. 2022 - Sep. 2022} % Date(s)
{
    \begin{cvitems}
        \item{A linear least squares problem with an ill-conditioned matrix was solved using L-BFGS, Thin QR factorization, Conjugate Gradient and Gradient Descent.}
        \item{\textbf{Technical skills:} MATLAB.}
    \end{cvitems}
}

%---------------------------------------------------------
\cventry
{Pathway analysis of disease's proteins} % Project Description
{Computational Health Laboratory} % Course
{\href{https://github.com/nikodallanoce/ComputationalHealthLaboratory}{\faGithubSquare Project repository}} % Location
{Apr. 2022 - May 2022} % Date(s)
{
    \begin{cvitems}
        \item{A protein-to-protein graph was constructed starting from a single protein that is primarily responsible for a disease. The correlation with all other proteins in the network was then determined.}
        \item{\textbf{Technical skills:} Python (pandas, seaborn), Jupyter Notebook, Git.}
    \end{cvitems}
}

%---------------------------------------------------------
\cventry
{Simulation of highway traffic via cellular automata} % Project Description
{Computational Models for Complex Systems} % Course
{\href{https://github.com/nikodallanoce/TrafficAutomata}{\faGithubSquare Project repository}} % Location
{May 2022 - May 2022} % Date(s)
{
    \begin{cvitems}
        \item{Simulated the flow of highway traffic in various scenarios using the cellular automata paradigm.}
        \item{\textbf{Technical skills:} Java, Git.}
    \end{cvitems}
}

%---------------------------------------------------------
\cventry
{Parallelization of a custom KNN algorithm} % Project Description
{Parallel and Distributed Systems: Paradigms and Models} % Course
{\href{https://github.com/RistoAle97/PDS}{\faGithubSquare Project repository}} % Location
{Jan. 2022 - Feb. 2022} % Date(s)
{
    \begin{cvitems}
        \item{A custom implementation of the KNN algorithm was parallelized by using both the standard library of C++ and the FastFlow library. The performance of both implementations was then compared.}
        \item{\textbf{technical skills:} C++, Python, Linux.}
    \end{cvitems}
}

%---------------------------------------------------------
\cventry
{Object recognition for an autonomous driving car} % Project Description
{Smart Applications} % Course
{\href{https://github.com/unipi-smartapp-2021/sensory-cone-detection}{\faGithubSquare Project repository}} % Location
{Nov. 2021 - Jan. 2022} % Date(s)
{
    \begin{cvitems}
        \item{Took part in a eight-person team responsible for developing an object recognition model for the stereocamera and lidar of an autonomous driving vehicle.}
        \item{\textbf{Technical skills:} Python (pandas, YOLOv3), Git.}
    \end{cvitems}
}

%---------------------------------------------------------
\cventry
{Analysis and study of tennis matches data} % Project Description
{Data Mining} % Course
{\href{https://github.com/nikodallanoce/DataMiningProject}{\faGithubSquare Project repository}} % Location
{Sep. 2021 - Jan. 2022} % Date(s)
{
    \begin{cvitems}
        \item {Worked on a dataset of more than 100k tennis matches. Initially the dataset was cleaned by removing missing or incorrect data. Subsequently, the performance of all players were analyzed in order to classify them into different categories. Finally, the results were presented in a human-readable format.}
        \item{\textbf{Technical skills:} Python (pandas, seaborn, numpy, scikit), Jupyter Notebook, Git.}
    \end{cvitems}
}

%---------------------------------------------------------
\cventry
{Comparison of different NMT models} % Project Description
{Human Language Technologies} % Course
{\href{https://github.com/nikodallanoce/NeuralMachineTranslation}{\faGithubSquare Project repository}} % Location
{Sep. 2021 - Dec. 2021} % Date(s)
{
    \begin{cvitems}
        \item{Compared various Neural Machine Translation models to evaluate their performance while modifying their decoder.}
        \item{\textbf{Technical skills:} Python (Tensorflow), Jupyter Notebook, Git.} 
    \end{cvitems}
}

%---------------------------------------------------------
\cventry
{Telegram bot for monitoring room temperature} % Project description
{Mobile and Cyber-Physical Systems} % Course
{\href{https://github.com/RistoAle97/BotTelegramMCPS}{\faGithubSquare Project repository}} % Location
{Mar. 2021 - Jun. 2021} % Date(s)
{
    \begin{cvitems}
        \item{Developed a Telegram bot in order to provide users with updates on selected rooms' data. The purpose of this was to inform shop owners if their refrigerator cells were experiencing issues and to determine if these rooms were compliant with HACCP regulations.}
        \item{\textbf{Technical skills:} Python, MongoDB, Java, Git}
    \end{cvitems}
}

%---------------------------------------------------------
\cventry
{Development of a neural network from scratch} % Project description
{Machine Learning} % Course
{\href{https://github.com/nikodallanoce/MLProject}{\faGithubSquare Project repository}} % Location
{Nov. 2020 - Jan. 2021} % Date(s)
{
    \begin{cvitems}
        \item{Built a basic library that handles the creation of layers for a neural network, as well as executing the forward and backward passes.}
        \item{\textbf{Technical skills:} Python (numpy), Git.}
    \end{cvitems}
}

%---------------------------------------------------------
\end{cventries}
